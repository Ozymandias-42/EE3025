\documentclass[journal,12pt,twocolumn]{IEEEtran}

\usepackage{setspace}
\usepackage{gensymb}
\singlespacing
\usepackage[cmex10]{amsmath}

\usepackage{amsthm}

\usepackage{mathrsfs}
\usepackage{txfonts}
\usepackage{stfloats}
\usepackage{bm}
\usepackage{cite}
\usepackage{cases}
\usepackage{subfig}

\usepackage{longtable}
\usepackage{multirow}

\usepackage{enumitem}
\usepackage{mathtools}
%\usepackage{steinmetz}
\usepackage{tikz}
\usepackage{circuitikz}
\usepackage{verbatim}
%\usepackage{tfrupee}
\usepackage[breaklinks=true]{hyperref}
\usepackage{graphicx}
\usepackage{tkz-euclide}

\usetikzlibrary{calc,math}
\usepackage{listings}
    \usepackage{color}                                            %%
    \usepackage{array}                                            %%
    \usepackage{longtable}                                        %%
    \usepackage{calc}                                             %%
    \usepackage{multirow}                                         %%
    \usepackage{hhline}                                           %%
    \usepackage{ifthen}                                           %%
    \usepackage{lscape}     
\usepackage{multicol}
\usepackage{chngcntr}

\DeclareMathOperator*{\Res}{Res}

\renewcommand\thesection{\arabic{section}}
\renewcommand\thesubsection{\thesection.\arabic{subsection}}
\renewcommand\thesubsubsection{\thesubsection.\arabic{subsubsection}}

\renewcommand\thesectiondis{\arabic{section}}
\renewcommand\thesubsectiondis{\thesectiondis.\arabic{subsection}}
\renewcommand\thesubsubsectiondis{\thesubsectiondis.\arabic{subsubsection}}


\hyphenation{op-tical net-works semi-conduc-tor}
\def\inputGnumericTable{}                                 %%

\lstset{
%language=C,
frame=single, 
breaklines=true,
columns=fullflexible
}
\begin{document}


\newtheorem{theorem}{Theorem}[section]
\newtheorem{problem}{Problem}
\newtheorem{proposition}{Proposition}[section]
\newtheorem{lemma}{Lemma}[section]
\newtheorem{corollary}[theorem]{Corollary}
\newtheorem{example}{Example}[section]
\newtheorem{definition}[problem]{Definition}

\newcommand{\BEQA}{\begin{eqnarray}}
\newcommand{\EEQA}{\end{eqnarray}}
\newcommand{\define}{\stackrel{\triangle}{=}}
\bibliographystyle{IEEEtran}
\raggedbottom
\setlength{\parindent}{0pt}
\providecommand{\mbf}{\mathbf}
\providecommand{\pr}[1]{\ensuremath{\Pr\left(#1\right)}}
\providecommand{\qfunc}[1]{\ensuremath{Q\left(#1\right)}}
\providecommand{\sbrak}[1]{\ensuremath{{}\left[#1\right]}}
\providecommand{\lsbrak}[1]{\ensuremath{{}\left[#1\right.}}
\providecommand{\rsbrak}[1]{\ensuremath{{}\left.#1\right]}}
\providecommand{\brak}[1]{\ensuremath{\left(#1\right)}}
\providecommand{\lbrak}[1]{\ensuremath{\left(#1\right.}}
\providecommand{\rbrak}[1]{\ensuremath{\left.#1\right)}}
\providecommand{\cbrak}[1]{\ensuremath{\left\{#1\right\}}}
\providecommand{\lcbrak}[1]{\ensuremath{\left\{#1\right.}}
\providecommand{\rcbrak}[1]{\ensuremath{\left.#1\right\}}}
\theoremstyle{remark}
\newtheorem{rem}{Remark}
\newcommand{\sgn}{\mathop{\mathrm{sgn}}}
\providecommand{\abs}[1]{\left\vert#1\right\vert}
\providecommand{\res}[1]{\Res\displaylimits_{#1}} 
\providecommand{\norm}[1]{\left\lVert#1\right\rVert}
\providecommand{\norm}[1]{\lVert#1\rVert}
\providecommand{\mtx}[1]{\mathbf{#1}}
\providecommand{\mean}[1]{E\left[ #1 \right]}
\providecommand{\fourier}{\overset{\mathcal{F}}{ \rightleftharpoons}}
%\providecommand{\hilbert}{\overset{\mathcal{H}}{ \rightleftharpoons}}
\providecommand{\system}{\overset{\mathcal{H}}{ \longleftrightarrow}}
	%\newcommand{\solution}[2]{\textbf{Solution:}{#1}}
\newcommand{\solution}{\noindent \textbf{Solution: }}
\newcommand{\cosec}{\,\text{cosec}\,}
\providecommand{\dec}[2]{\ensuremath{\overset{#1}{\underset{#2}{\gtrless}}}}
\newcommand{\myvec}[1]{\ensuremath{\begin{pmatrix}#1\end{pmatrix}}}
\newcommand{\mydet}[1]{\ensuremath{\begin{vmatrix}#1\end{vmatrix}}}
\numberwithin{equation}{subsection}
\makeatletter
\@addtoreset{figure}{problem}
\makeatother
\let\StandardTheFigure\thefigure
\let\vec\mathbf
\renewcommand{\thefigure}{\theproblem}
\def\putbox#1#2#3{\makebox[0in][l]{\makebox[#1][l]{}\raisebox{\baselineskip}[0in][0in]{\raisebox{#2}[0in][0in]{#3}}}}
     \def\rightbox#1{\makebox[0in][r]{#1}}
     \def\centbox#1{\makebox[0in]{#1}}
     \def\topbox#1{\raisebox{-\baselineskip}[0in][0in]{#1}}
     \def\midbox#1{\raisebox{-0.5\baselineskip}[0in][0in]{#1}}
\vspace{3cm}
\title{Assignment 1}
\author{Abhishek Shetkar - EE18BTECH1102}
\maketitle
\newpage
\bigskip
\renewcommand{\thefigure}{\theenumi}
\renewcommand{\thetable}{\theenumi}

%
The latex file can be found at: 
%
\begin{lstlisting}
https://github.com/Ozymandias-42/EE3025/A1
\end{lstlisting}
\section{Problem}
(5.3) The system h(n) is said to be stable if \\
\begin{align}
\sum_{n=-\infty}^{\infty} \abs{h(n)} < \infty \\
\end{align}
Is the system defined by (3.2) stable for impulse response in (5.1)?
\section{Solution}
\textbf{Impulse response} : Impulse response is the systems response to unit impulse signal.
If the system is stable for unit impulse, due to it's LTI nature, it will be stable for all bounded inputs.
\\
\\  
\\
\textbf{Direct solution (without using Z-transform)}
\\
\\
Given stability condition
\begin{align}
\sum_{n=-\infty}^{\infty} \abs{h(n)}  < \infty\\
\end{align}

Given $h(n) = \brak{\frac{-1}{2}}^{n}u(n) + \brak{\frac{-1}{2}}^{n-2}u(n-2)$ 
\\
\\
To check whether for given $h(n)$
\begin{align}
\sum_{n=-\infty}^{\infty} \abs{h(n)} < \infty
\end{align}
Substituting given h(n) on LHS
\begin{align}
\sum_{n=-\infty}^{\infty} \abs{h(n)} = \sum_{n=-\infty}^{\infty} \abs{\brak{\frac{-1}{2}}^{n}u(n) + \brak{\frac{-1}{2}}^{n-2}u(n-2)}\\
\end{align}
Observe that, for odd values of n, the individual terms in the summation will be negative but due to the modulus, their positive value will be evaluated. Hence, in essence, the $\brak{\frac{-1}{2}}$ will become $\brak{\frac{1}{2}}$\\
Therefore, our expression becomes:
\begin{align}
\sum_{n=-\infty}^{\infty} \brak{\frac{1}{2}}^{n}u(n) + \brak{\frac{1}{2}}^{n-2}u(n-2)\\
=\sum_{n=0}^{\infty}\brak{\frac{1}{2}}^{n} + \sum_{n=2}^{\infty}\brak{\frac{1}{2}}^{n-2}\\
\end{align}
Observe that the second summation is the same as first summation if we take an $\alpha=n-2$\\

Now,Using Geometric progression formula 
\begin{align}
\sum_{n=0}^{\infty}\brak{\frac{1}{a}}^{n} = \frac{1}{1-\frac{1}{a}}
\end{align}
\begin{align}
\sum_{n=0}^{\infty}\brak{\frac{1}{2}}^{n} + \sum_{\alpha=0}^{\infty}\brak{\frac{1}{2}}^{\alpha} = 2 . \frac{1}{1-\frac{1}{2}}=4\\ 
\end{align}
Thus, we have proved that;
\begin{align}
\sum_{n=-\infty}^{\infty} \abs{h(n)} = 4\\
\implies \sum_{n=-\infty}^{\infty} \abs{h(n)} < \infty\\
\end{align}
Thus, the impulse response of the system satisfies the given condition.\\
Hence, the system is stable.
\end{document}